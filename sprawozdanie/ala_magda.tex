\documentclass[10pt,a4paper]{article}

% ------------------------- PREAMBUŁA -------------------   % stała ścieżka względna do katalogu z  obrazkami.
\usepackage{graphicx}
%\input{settings/packages}  
%\graphicspath{{images/}} 
\usepackage{graphicx}
\usepackage{subcaption}
\usepackage[polish]{babel}
\usepackage[T1]{fontenc}
\usepackage[utf8]{inputenc}
\usepackage{amsmath, amsfonts, amssymb}
\usepackage{booktabs}
\usepackage[top=2.5cm, bottom=2.5cm, left=2cm, right=2cm]{geometry}
\usepackage{textcomp}
\usepackage{gensymb}
\usepackage{textgreek}
\usepackage{geometry}
\usepackage{pdflscape}
\usepackage{pdfpages}
\usepackage{hyperref}
\usepackage{xcolor}

% METADATA
% DOCUMENT METADATA
\newcommand{\logoGIK}{WGiK-znak.png}
\newcommand{\authorName}{Alicja Łubianka \\ Numer indeksu: 325786, grupa 3 \\ e-mail: 01179176@pw.edu.pl\\ Magdalena Sternik \\ Numer indeksu: 325835, grupa 3 \\ e-mail: magdy}

\newcommand{\titeReport}{Transformacje miedzy różnymi układami geodezyjnymi} % <<< here insert short title in the food
\newcommand{\titleLecture}{INFORMATYKA GEODEZYJNA II\\ sem. IV, ćwiczenia, rok akad. 2023-2024} % <<< insert title of presentation
\newcommand{\kind}{report}
\newcommand{\supervisor}{....}
\newcommand{\gikweb}{\href{www.gik.pw.edu.pl}{www.gik.pw.edu.pl}}
\newcommand{\institut}{Zakład Geodezji Wyższej i Astronomii}
\newcommand{\faculty}{Wydział Geodezji i Kartografii}
\newcommand{\university}{Politechnika Warszawska}
\newcommand{\city}{Warszawa}
\newcommand{\thisyear}{2023}
%\date{}
% PDF METADATA
\pdfinfo
{
	/Title       (GIK PW)
	/Creator     (TeX)
	/Author      (Imię Nazwisko)
}
\begin{document}
	\begin{center} 
		\rule{\textwidth}{.5pt} \\
		\vspace{1.0cm}
		\includegraphics[width=.4\paperwidth]{\logoGIK}
		\vspace{0.5cm} \\
		\Large \textsc{\titeReport}
		\vspace{0.5cm} \\  
		\large \textsc{\titleLecture}
		\vspace{0.5cm}\\
		\textsc{\authorName}  \\
		\textsc{\faculty}, \textsc{\university}  \\ 
		\city, \today
	\end{center}
	\newpage
	\tableofcontents
	\newpage
	\section{Cel ćwiczenia}
	W ramach ćwiczenia opracowano skrypt w języku Python w postaci klasy zawierającej metody służące do transformacji współrzędnych pomiędzy układem kartezjańskim (x, y, z) a geodezyjnym ($\phi$, $\lambda$, H). Aby dokonać transformacji pomiędzy układami należy zastosować odpowiednie algorytmy. Poniżej przedstawiona jest lista utworzonych algorytmów:
	\begin{itemize}
		\item XYZ ( geocentryczne) -> BLH ( elipsoidalne fi, lambda, h)
		\item BLH -> XYZ 
		\item XYZ -> NEU ( topocentryczne northing, easting, up)
		\item BL ( GRS80, WGS84, ew. Krasowski) -> PL2000
		\item BL ( GRS80, WGS84, ew. Krasowski) -> PL1992
		
	\end{itemize}
	\section{Wykorzystane narzędzia i materiały potrzebne do replikacji ćwiczenia}
	
	\subsection{Wybrany język programowania i interpreter Spyder}
	

		
	Do napisania skryptu tego ćwiczenia posłużył nam język programowania Python, a za środowisko odpowiadał Spyder zawierający edytor kodu, interpreter, konsolę, a także inne funkcje.
	 

	\subsection{System operacyjny}
	
	Plik został utworzony w systemie operacyjnym Microsoft ( Windows 11).
	
	\subsection{Potrzebne biblioteki i pliki}
	
	Do wykonania ćwiczenia należy użyć następujących bibliotek:
	\begin{enumerate}

	\item Numpy - biblioteka w języku Python służąca do obliczeń numerycznych i analizy danych. Zapewnia narzędzia do pracy z wielowymiarowymi tablicami danych oraz operacji matematycznych i statystycznych na tych tablicach. Numpy nie jest wbudowany w Pythona, lecz jest dostarczany z Anacondą, co ułatwia jego dostępność.
	\item Argparse - biblioteka w języku Python do parsowania argumentów linii poleceń. Jest częścią standardowej biblioteki Pythona, co oznacza, że jest wbudowana w standardową instalację Anacondy.
	\item Os - biblioteka standardowa w języku Python zapewniająca interfejs do operacji na systemie operacyjnym, takich jak dostęp do plików, zarządzanie procesami, zmiana katalogu roboczego, itp.	
		
	\end{enumerate}
	\section{Przebieg ćwiczenia}
	
	\subsection{Utworzenie klasy Transformacja}
	Utworzono klasę Transformację oraz konstruktor __init__, w którym przekazywane są parametry elipsoidy ( a, e2), niezbędne do wykonywania obliczeń. Ważne, aby pamiętać, gdy się odwołujemy do powyższych parametrów, musimy zastosować self. Dodatkowo, pokazano sytuację, gdy podana zostanie nieprawidłowa nazwa elipsoidy, co skutkuje wyświetleniem błędu. 
	
	\subsection{Algorytm hirvonena} 
	Algorytm Hirvonena przekształca współrzędne kartezjańskie ( X,Y,Z) na współrzędne geodezyjne ($\phi$, $\lambda$, H). W funkcji implementującej ten algorytm, używana jest pętla while, która wykonuje odpowiednią liczbę iteracji, aby uzyskać dokładność na poziomie 1 mm. Dodatkowo zaimplementowany został warunek „output” za pomocą instrukcji warunkowych elif, if, else, pozwalający na wybór formatu wyniku. Do jednej z metod przekształceniowej ( stopnie, minuty sekundy) powstał dodatkowy algorytm „dms”. Prócz tego utworzono funkcję „np”, obliczającą promień przekroju w pierwszym wertykale. Wyniki działania transformacji zostały zweryfikowane z wynikami uzyskanymi w zaliczonym sprawozdaniu z poprzedniego semestru.
	
	\subsection{flh2XYZ}
	Transformacja flh konwertuje współrzędne geodezyjne (φ, λ, h) na współrzędne kartezjańskie (x,y,z). W implementacji tej transformacji zastosowano trzy wzory, z których każdy odpowiada jednej ze współrzędnych kartezajńskich. Dodatkowo wykorzystana została funkcja „Np”. Wyniki działania transformacji zostały zweryfikowane z wynikami uzyskanymi w zaliczonym sprawozdaniu z poprzedniego semestru.
	
	 

	\subsection{flh2PL1992}
	Transormacja fl21992 przekształca współrzędne geodezyjne (φ, λ) na współrzędne w układzie 1992 ( x1992, y1992). W implementacji tej transformacji dodano warunek ograniczający współrzędne φ i λ do obszaru wyłącznie Polski, aby uniknąć błędów. Następnie obliczono wartości x1992, y1992 przy użyciu odpowiednich wzorów. Wyniki działania transformacji zostały zweryfikowane z wynikami uzyskanymi w zaliczonym sprawozdaniu z poprzedniego semestru.  

	\subsection{flh2PL2000} 
	Transformacja fl2PL2000 konwertuje współrzędne geodezyjne (φ, λ) na współrzędne w układzie 2000 ( x2000, y2000). Implementacja tej transformacji przebiega niemal identycznie jak tranformacja fl2PL1992, z wyjątkiem zmiany skali oraz ustaleniem odpowiedniej strefy ( 5,6,7,8) dla naszych danych. Po zamianie współrzędnych otrzymano x2000, y2000. Wyniki działania transformacji zostały zweryfikowane z wynikami uzyskanymi w zaliczonym sprawozdaniu z poprzedniego semestru. 

	
	\subsection{xyz2neu} 
	Transformacja xyz2neu przekształca współrzędne kartezjańskie ( x,y,z ) do układu neu. W implementacji tej transformacji stworzono trzy definicje. Pierwsza dotyczyła mecierzy obrotu (renu), druga obliczała macierz różnic między dwoma punktami, a trzecia obliczała macierz neu. Pierwsza kolumna ( odpowiadała współrzędnej n, druga współrzędnej e, a trzecia u. Wyniki działania transformacji zostały zweryfikowane z wynikami uzyskanymi w zaliczonym sprawozdaniu z poprzedniego semestru. 
	\newpage
	\subsection{Wczytywanie i zapisywanie pliku}
	Aby odczytać i zapisać plik, stworzono trzy funkcje. Pierwsza służy do odczytu pliku w formacie txt, a druga transformuje zmienne głównie do stringów, aby wszystkie miały taką samą długość. Do tego celu użyto funkcji zamian float na string: 

	\begin{itemize}
		\item f2s;
		\item f2s_fl;
		\item f2s_rad,

	\end{itemize}
	które działają na tej samej zasadzie, czyli dodają spację przed liczbą za pomocą pętli while, a warunek kończy się, gdy string osiągnie odpowiednią długość. Trzecia funkcja zapisuje plik w postaci tabelki z nagłówkiem. Dla punktów spoza Polski w wynikach ( x1992, y1992, x2000, y2000) stosowane są myślniki.

	\subsection{Kalkulator transformacji i zapis ich wyników do Kalkulatora}
	W celu wykonania transformacji współrzędnych do różnych układów, został stworzony kalkulator o nazwie „kalkulator_xyz2reszta.py. Korzysta z importowanych transformacji z pliku głównego oraz biblioteki argparse, wykorzystując ArgumentParser. 
	
	tutaj ala uzupełnij !!!

	($\phi$, $\lambda$, H) 

	
	\subsection{Dodanie możliwości wczytania pliku w argparse}
	Dodatkowo w głównym pliku dodaliśmy funkcję umożliwiającą wczytywanie i zapisanie pliku wynikowego za pomocą biblioteki argparse. Dzięki temu użytkownik może teraz użytkownik może teraz wczytać plik z wiersza poleceń. 

	

		
\end{document}

