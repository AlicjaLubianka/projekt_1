\documentclass[10pt,a4paper]{article}

% ------------------------- PREAMBUŁA -------------------   % stała ścieżka względna do katalogu z  obrazkami.
\usepackage{graphicx}
%\input{settings/packages}  
%\graphicspath{{images/}} 
\usepackage{graphicx}
\usepackage{subcaption}
\usepackage[polish]{babel}
\usepackage[T1]{fontenc}
\usepackage[utf8]{inputenc}
\usepackage{amsmath, amsfonts, amssymb}
\usepackage{booktabs}
\usepackage[top=2.5cm, bottom=2.5cm, left=2cm, right=2cm]{geometry}
\usepackage{textcomp}
\usepackage{gensymb}
\usepackage{textgreek}
\usepackage{geometry}
\usepackage{pdflscape}
\usepackage{pdfpages}
\usepackage{hyperref}
\usepackage{xcolor}

% METADATA
% DOCUMENT METADATA
\newcommand{\logoGIK}{WGiK-znak.png}
\newcommand{\authorName}{Alicja Łubianka \\ Numer indeksu: 325786, grupa 3 \\ e-mail: 01179176@pw.edu.pl\\ Magdalena Sternik \\ Numer indeksu: 325835, grupa 3 \\ e-mail: magdy}

\newcommand{\titeReport}{Transformacje miedzy różnymi układami geodezyjnymi} % <<< here insert short title in the food
\newcommand{\titleLecture}{INFORMATYKA GEODEZYJNA II\\ sem. IV, ćwiczenia, rok akad. 2023-2024} % <<< insert title of presentation
\newcommand{\kind}{report}
\newcommand{\supervisor}{....}
\newcommand{\gikweb}{\href{www.gik.pw.edu.pl}{www.gik.pw.edu.pl}}
\newcommand{\institut}{Zakład Geodezji Wyższej i Astronomii}
\newcommand{\faculty}{Wydział Geodezji i Kartografii}
\newcommand{\university}{Politechnika Warszawska}
\newcommand{\city}{Warszawa}
\newcommand{\thisyear}{2023}
%\date{}
% PDF METADATA
\pdfinfo
{
	/Title       (GIK PW)
	/Creator     (TeX)
	/Author      (Imię Nazwisko)
}
\begin{document}
	\begin{center} 
		\rule{\textwidth}{.5pt} \\
		\vspace{1.0cm}
		\includegraphics[width=.4\paperwidth]{\logoGIK}
		\vspace{0.5cm} \\
		\Large \textsc{\titeReport}
		\vspace{0.5cm} \\  
		\large \textsc{\titleLecture}
		\vspace{0.5cm}\\
		\textsc{\authorName}  \\
		\textsc{\faculty}, \textsc{\university}  \\ 
		\city, \today
	\end{center}
	\newpage
	\tableofcontents
	\newpage
	\section{Cel ćwiczenia}
	W ramach ćwiczenia opracowano skrypt w języku Python w postaci klasy zawierającej metody służące do transformacji współrzędnych pomiędzy układem kartezjańskim (x, y, z) a geodezyjnym ($\phi$, $\lambda$, H). W implementacji wykorzystano odpowiednie algorytmy przekształceń współrzędnych, zapewniające dokładność i poprawność operacji.
	\begin{itemize}
		\item hirvonen(xyz2flh)
		\item flh2xyz
		\item flh2PL92
		\item flh2PL20
		\item xyz2neu 
		
	\end{itemize}
	\section{Wykorzystane narzędzia i materiały potrzebne do replikacji ćwiczenia}
	
	\subsection{Wybrany język programowania i interpreter Spyder}
	
	\begin{itemize}
		\item 
		Python - język programowania, w którym został napisany skrypt ćwiczenia.
		\item Spyder - to środowisko programistyczne dla języka Python, które oferuje edytor kodu, interpreter, konsolę i wiele innych funkcjonalności.
		\item Najlepiej pobrać Spydera poprzez Anacondę, która domyślnie zawiera środowisko programistyczne Spyder 
	\end{itemize}
	
	\subsection{System operacyjny}
	
	Ten skrypt został napisany w systemie operacyjnym Microsoft (Windows 11).
	
	\subsection{Potrzebne biblioteki i pliki}
	
	Do wykonania ćwiczenia należy użyć następujących bibliotek:
	\begin{enumerate}

	\item Numpy - biblioteka w języku Python służąca do obliczeń numerycznych i analizy danych. Zapewnia narzędzia do pracy z wielowymiarowymi tablicami danych oraz operacji matematycznych i statystycznych na tych tablicach. Numpy nie jest wbudowany w Pythona, lecz jest dostarczany z Anacondą, co ułatwia jego dostępność.
	\item Argparse - biblioteka w języku Python do parsowania argumentów linii poleceń. Jest częścią standardowej biblioteki Pythona, co oznacza, że jest wbudowana w standardową instalację Anacondy.
	\item Os - biblioteka standardowa w języku Python zapewniająca interfejs do operacji na systemie operacyjnym, takich jak dostęp do plików, zarządzanie procesami, zmiana katalogu roboczego, itp.	
	\item Pytest - biblioteka w języku Python służąca do testowania kodu źródłowego. Umożliwia łatwe i elastyczne pisanie testów. Pytest nie jest wbudowany ani w standardową instalację Pythona, ani w dystrybucję pakietów Anaconda, ale można go zainstalować za pomocą menedżera pakietów pip.
	\item Tkinter - biblioteka graficzna dla języka programowania Python. Umożliwia tworzenie interfejsów graficznych użytkownika (GUI) dla programów Python. Tkinter jest dostępny w standardowej bibliotece Pythona i łatwo dostępny na większości platform.
		
	\end{enumerate}
	
.... wpisac tu potrzebne pliki

		
\end{document}

